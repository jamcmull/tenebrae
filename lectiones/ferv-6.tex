\begin{tabular}{p{8cm} | p{8cm}}
Quóniam vidi iniquitátem, et contradictiónem in civitáte. Atténde glóriam crucis ipsíus. Jam in fronte regum crux illa fixa est, cui inimíci insultavérunt. Efféctus probávit virtútem: dómuit orbem non ferro, sed ligno. Lignum crucis contuméliis dignum visum est inimícis, et ante ipsum lignum stantes caput agitábant, et dicébant: Si Fílius Dei est, descéndat de cruce. Extendébat ille manus suas ad pópulum non credéntem, et contradicéntem. Si enim justus est, qui ex fide vivit; iníquus est, qui non habet fidem. Quod ergo hic ait, iniquitátem: perfídiam intéllege. Vidébat ergo Dóminus in civitáte iniquitátem et contradictiónem, et extendébat manus suas ad pópulum non credéntem et contradicéntem: et tamen et ipsos exspéctans dicébat: Pater, ignósce illis, quia nésciunt quid fáciunt.
& \textit{We have seen iniquity and strife in the city. Behold, the glory of the Cross. That Cross which was the object of the insults of God's enemies, is established now above the brows of kings. The end hath shown the measure of its power: it hath conquered the world, not by the sword, but by its wood. The enemies of God thought the Cross a meet object of insult and ridicule, yea, they stood before it, wagging their heads and saying: If He be the Son of God, let Him come down from the Cross! And He stretched forth His Hands unto a disobedient and gainsaying people. If he is just which liveth by faith, he is unjust that hath not faith. Therefore where is written iniquity we may understand unbelief. The Lord therefore saith that He saw iniquity and strife in the city, and that He stretched forth His Hands unto that disobedient and gainsaying people, and, disobedient and gainsaying as they were, He was hungry for their salvation, and said: Father, forgive them, for they know not what they do.}
\end{tabular}