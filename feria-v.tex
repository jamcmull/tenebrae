% !TEX TS-program = lualatex
% !TEX encoding = UTF-8

\documentclass[11pt, twoside]{report}

\usepackage{fontspec}
\usepackage[utf8]{inputenc}
\usepackage[bitstream-charter]{mathdesign}
\usepackage{bbding}
\usepackage{ragged2e}
\usepackage{parskip}
\usepackage{titlesec}
\usepackage{longtable}
\usepackage[margin=1in]{geometry}

\usepackage[autocompile]{gregoriotex}

\titleformat{\chapter}[block]{\huge\scshape\filcenter}{}{1ex}{}
\titleformat{\section}[block]{\LARGE\scshape\filcenter}{}{1ex}{}
\titleformat{\subsection}[block]{\Large\scshape\filcenter}{}{1ex}{}
\titleformat{\subsubsection}[block]{\large\bfseries\filcenter}{}{1ex}{}

\newenvironment{versicles}{\par\leavevmode\parskip=0pt\nopagebreak}{}

\begin{document}

\chapter*{{\Large Feria Quinta\\} In C\oe na Domini.}
\section*{Ad Matutinum.}

\subsection*{In primo Nocturno.}

\subsubsection*{Psalmus 68.}
\gresetinitiallines{1}
\gregorioscore{antiphonae/ferv-m1}
\gresetinitiallines{0}
\gregorioscore{psalmi/incipit/68-8c}
\begin{longtable}{@{\hskip0pt} p{10cm} | p{6cm} @{\hskip0pt}}
2.\enspace Infíxus sum in limo pro\textbf{fún}di:~* et non \textit{est} \textit{sub}\textbf{stán}tia.
 & \textit{\small I stick fast in the mire of the deep: and there is no sure standing.
}\\
3.\enspace Veni in altitúdinem \textbf{ma}ris:~* et tempés\textit{tas} \textit{de}\textbf{mér}sit me.
 & \textit{\small I am come into the depth of the sea: and a tempest hath overwhelmed me.
}\\
4.\enspace Laborávi clamans, raucæ factæ sunt fauces \textbf{me}æ:~* defecérunt óculi mei, dum spero in \textit{De}\textit{um} \textbf{me}um.
 & \textit{\small I have laboured with crying; my jaws are become hoarse: my eyes have failed, whilst I hope in my God.
}\\
5.\enspace Multiplicáti sunt super capíllos cápitis \textbf{me}i,~* qui odé\textit{runt} \textit{me} \textbf{gra}tis.
 & \textit{\small They are multiplied above the hairs of my head, who hate me without cause.
}\\
6.\enspace Confortáti sunt qui persecúti sunt me inimíci mei in\textbf{jús}te:~* quæ non rápui, tunc \textit{ex}\textit{sol}\textbf{vé}bam.
 & \textit{\small My enemies are grown strong who have wrongfully persecuted me: then did I pay that which I took not away.
}\\
7.\enspace Deus, tu scis insipiéntiam \textbf{me}am:~* et delícta mea a te non \textit{sunt} \textit{abs}\textbf{cón}dita.
 & \textit{\small O God, thou knowest my foolishness; and my offences are not hidden from thee.
}\\
8.\enspace Non erubéscant in me qui exspéctant te, \textbf{Dó}mine,~* Dómi\textit{ne} \textit{vir}\textbf{tú}tum
 & \textit{\small Let not them be ashamed for me, who look for thee, O Lord, the Lord of hosts.
}\\
9.\enspace Non confundántur \textbf{su}per me~* qui quærunt te, \textit{De}\textit{us} \textbf{Is}raël.
 & \textit{\small Let them not be confounded on my account, who seek thee, O God of Israel.
}\\
10.\enspace Quóniam propter te sustínui op\textbf{pró}brium:~* opéruit confúsio fá\textit{ci}\textit{em} \textbf{me}am.
 & \textit{\small Because for thy sake I have borne reproach; shame hath covered my face.
}\\
11.\enspace Extráneus factus sum frátribus \textbf{me}is,~* et peregrínus fíliis \textit{ma}\textit{tris} \textbf{me}æ.
 & \textit{\small I am become a stranger to my brethren, and an alien to the sons of my mother.
}\\
12.\enspace Quóniam zelus domus tuæ com\textbf{é}dit me:~* et oppróbria exprobrántium tibi ceci\textit{dé}\textit{runt} \textbf{su}per me.
 & \textit{\small For the zeal of thy house hath eaten me up: and the reproaches of them that reproached thee are fallen upon me.
}\\
13.\enspace Et opérui in jejúnio ánimam \textbf{me}am:~* et factum est in oppró\textit{bri}\textit{um} \textbf{mi}hi.
 & \textit{\small And I covered my soul in fasting: and it was made a reproach to me.
}\\
14.\enspace Et pósui vestiméntum meum ci\textbf{lí}cium:~* et factus sum illis \textit{in} \textit{pa}\textbf{rá}bolam.
 & \textit{\small And I made haircloth my garment: and I became a byword to them.
}\\
15.\enspace Advérsum me loquebántur, qui sedébant in \textbf{por}ta:~* et in me psallébant qui bi\textit{bé}\textit{bant} \textbf{vi}num.
 & \textit{\small They that sat in the gate spoke against me: and they that drank wine made me their song.
}\\
16.\enspace Ego vero oratiónem meam ad te, \textbf{Dó}mine:~* tempus beneplá\textit{ci}\textit{ti}, \textbf{De}us.
 & \textit{\small But as for me, my prayer is to thee, O Lord; for the time of thy good pleasure, O God.
}\\
17.\enspace In multitúdine misericórdiæ tuæ ex\textbf{áu}di me,~* in veritáte sa\textit{lú}\textit{tis} \textbf{tu}æ:
 & \textit{\small In the multitude of thy mercy hear me, in the truth of thy salvation.
}\\
18.\enspace Eripe me de luto, ut non in\textbf{fí}gar:~* líbera me ab iis, qui odérunt me, et de profún\textit{dis} \textit{a}\textbf{quá}rum.
 & \textit{\small Draw me out of the mire, that I may not stick fast: deliver me from them that hate me, and out of the deep waters.
}\\
19.\enspace Non me demérgat tempéstas aquæ,~\GreDagger\ neque absórbeat me pro\textbf{fún}dum:~* neque úrgeat super me púte\textit{us} \textit{os} \textbf{su}um.
 & \textit{\small Let not the tempest of water drown me, nor the deep swallow me up: and let not the pit shut her mouth upon me.
}\\
20.\enspace Exáudi me, Dómine, quóniam benígna est misericórdia \textbf{tu}a:~* secúndum multitúdinem miseratiónum tuárum ré\textit{spi}\textit{ce} \textbf{in} me.
 & \textit{\small Hear me, O Lord, for thy mercy is kind; look upon me according to the multitude of thy tender mercies.
}\\
21.\enspace Et ne avértas fáciem tuam a púero \textbf{tu}o:~* quóniam tríbulor, velóci\textit{ter} \textit{ex}\textbf{áu}di me.
 & \textit{\small And turn not away thy face from thy servant: for I am in trouble, hear me speedily.
}\\
22.\enspace Inténde ánimæ meæ, et líbera \textbf{e}am:~* propter inimícos meos \textit{é}\textit{ri}\textbf{pe} me.
 & \textit{\small Attend to my soul, and deliver it: save me because of my enemies.
}\\
23.\enspace Tu scis impropérium meum, et confusiónem \textbf{me}am,~* et reverén\textit{ti}\textit{am} \textbf{me}am.
 & \textit{\small Thou knowest my reproach, and my confusion, and my shame.
}\\
24.\enspace In conspéctu tuo sunt omnes qui tríbu\textbf{lant} me:~* impropérium exspectávit cor meum, \textit{et} \textit{mi}\textbf{sé}riam.
 & \textit{\small In thy sight are all they that afflict me; my heart hath expected reproach and misery.
}\\
25.\enspace Et sustínui qui simul contristarétur, et non \textbf{fu}it:~* et qui consolarétur, et \textit{non} \textit{in}\textbf{vé}ni.
 & \textit{\small And I looked for one that would grieve together with me, but there was none: and for one that would comfort me, and I found none.
}\\
26.\enspace Et dedérunt in escam \textbf{me}am fel:~* et in siti mea potavérunt \textit{me} \textit{a}\textbf{cé}to.
 & \textit{\small And they gave me gall for my food, and in my thirst they gave me vinegar to drink.
}\\
27.\enspace Fiat mensa eórum coram ipsis in \textbf{lá}queum,~* et in retributiónes, \textit{et} \textit{in} \textbf{scán}dalum.
 & \textit{\small Let their table become as a snare before them, and a recompense, and a stumblingblock.
}\\
28.\enspace Obscuréntur óculi eórum ne \textbf{ví}deant:~* et dorsum eórum sem\textit{per} \textit{in}\textbf{cúr}va.
 & \textit{\small Let their eyes be darkened that they see not; and their back bend thou down always.
}\\
29.\enspace Effúnde super eos iram \textbf{tu}am:~* et furor iræ tuæ compre\textit{hén}\textit{dat} \textbf{e}os.
 & \textit{\small Pour out thy indignation upon them: and let thy wrathful anger take hold of them.
}\\
30.\enspace Fiat habitátio eórum de\textbf{sér}ta:~* et in tabernáculis eórum non sit \textit{qui} \textit{in}\textbf{há}bitet.
 & \textit{\small Let their habitation be made desolate: and let there be none to dwell in their tabernacles.
}\\
31.\enspace Quóniam quem tu percussísti, perse\textbf{cú}ti sunt:~* et super dolórem vúlnerum meórum \textit{ad}\textit{di}\textbf{dé}runt.
 & \textit{\small Because they have persecuted him whom thou hast smitten; and they have added to the grief of my wounds.
}\\
32.\enspace Appóne iniquitátem super iniquitátem e\textbf{ó}rum:~* et non intrent in justí\textit{ti}\textit{am} \textbf{tu}am.
 & \textit{\small Add thou iniquity upon their iniquity: and let them not come into thy justice.
}\\
33.\enspace Deleántur de libro vi\textbf{vén}tium:~* et cum justis \textit{non} \textit{scri}\textbf{bán}tur.
 & \textit{\small Let them be blotted out of the book of the living; and with the just let them not be written.
}\\
34.\enspace Ego sum pauper et \textbf{do}lens:~* salus tua, De\textit{us}, \textit{su}\textbf{scé}pit me.
 & \textit{\small I am poor and sorrowful: thy salvation, O God, hath set me up.
}\\
35.\enspace Laudábo nomen Dei cum \textbf{cán}tico:~* et magnificábo e\textit{um} \textit{in} \textbf{lau}de:
 & \textit{\small I will praise the name of God with a canticle: and I will magnify him with praise.
}\\
36.\enspace Et placébit Deo super vítulum no\textbf{vél}lum:~* córnua producén\textit{tem} \textit{et} \textbf{ún}gulas.
 & \textit{\small And it shall please God better than a young calf, that bringeth forth horns and hoofs.
}\\
37.\enspace Vídeant páuperes et læ\textbf{tén}tur:~* qu\'{\ae}rite Deum, et vivet á\textit{ni}\textit{ma} \textbf{ves}tra.
 & \textit{\small Let the poor see and rejoice: seek ye God, and your soul shall live.
}\\
38.\enspace Quóniam exaudívit páuperes \textbf{Dó}minus:~* et vinctos suos \textit{non} \textit{de}\textbf{spé}xit.
 & \textit{\small For the Lord hath heard the poor: and hath not despised his prisoners.
}\\
39.\enspace Laudent illum cæli et \textbf{ter}ra,~* mare et ómnia reptíli\textit{a} \textit{in} \textbf{e}is.
 & \textit{\small Let the heavens and the earth praise him; the sea, and every thing that creepeth therein.
}\\
40.\enspace Quóniam Deus salvam fáciet \textbf{Si}on:~* et ædificabúntur civi\textit{tá}\textit{tes} \textbf{Ju}da.
 & \textit{\small For God will save Sion, and the cities of Juda shall be built up.
}\\
41.\enspace Et inhabitábunt \textbf{i}bi,~* et hereditáte ac\textit{quí}\textit{rent} \textbf{e}am.
 & \textit{\small And they shall dwell there, and acquire it by inheritance.
}\\
42.\enspace Et semen servórum ejus possidébit \textbf{e}am:~* et qui díligunt nomen ejus, habitá\textit{bunt} \textit{in} \textbf{e}a. & \textit{\small And the seed of his servants shall possess it; and they that love his name shall dwell therein.}\\
\end{longtable}

\subsubsection*{Psalmus 69.}
\gresetinitiallines{1}
\gregorioscore{antiphonae/ferv-m2}
\gresetinitiallines{0}
\gregorioscore{psalmi/incipit/69-8c}
\begin{longtable}{p{10cm} | p{6cm}}
2. Confundántur et revere\textbf{án}tur,~* qui quærunt á\textit{ni}\textit{mam} \textbf{me}am.
 & \textit{\small Let them be confounded and ashamed * that seek my soul:
}\\
3. Avertántur retrórsum, et eru\textbf{bés}cant,~* qui volunt \textit{mi}\textit{hi} \textbf{ma}la.
 & \textit{\small Let them be turned backward, and blush for shame * that desire evils to me:
}\\
4. Avertántur statim erube\textbf{scén}tes,~* qui dicunt mihi: \textit{Eu}\textit{ge}, \textbf{eu}ge.
 & \textit{\small Let them be presently turned away blushing for shame * that say to me: ’T is well, ’t is well.
}\\
5. Exsúltent et læténtur in te omnes qui \textbf{quæ}runt te,~* et dicant semper: Magnificétur Dóminus: qui díligunt salu\textit{tá}\textit{re} \textbf{tu}um.
 & \textit{\small Let all that seek thee rejoice and be glad in thee; * and let such as love thy salvation say always: The Lord be magnified.
}\\
6. Ego vero egénus, et \textbf{pau}per sum:~* Deus, \textit{ád}\textit{ju}\textbf{va} me.
 & \textit{\small But I am needy and poor; * O God, help me.
}\\
7. Adjútor meus, et liberátor meus \textbf{es} tu:~* Dómine, \textit{ne} \textit{mo}\textbf{ré}ris. & \textit{\small Thou art my helper and my deliverer: * O Lord, make no delay.}\\
\end{longtable}

\subsubsection*{Psalmus 70.}
\gresetinitiallines{1}
\gregorioscore{antiphonae/ferv-m3}
\gresetinitiallines{0}
\gregorioscore{psalmi/incipit/70-8c}
\begin{longtable}{@{\hskip0pt} p{9.5cm} | p{6.5cm} @{\hskip0pt}}
2.\enspace Inclína ad me aurem \textbf{tu}am,~* \textit{et} \textbf{sal}va me.
 & \textit{\small Incline thy ear unto me, * and save me.
}\\
3.\enspace Esto mihi in Deum protectórem, et in locum mu\textbf{ní}tum:~* ut sal\textit{vum} \textit{me} \textbf{fá}cias.
 & \textit{\small Be thou unto me a God, a protector, and a place of strength: * that thou mayst make me safe.
}\\
4.\enspace Quóniam firmaméntum \textbf{me}um,~* et refúgium \textit{me}\textit{um} \textbf{es} tu.
 & \textit{\small For thou art my firmament * and my refuge.
}\\
5.\enspace Deus meus, éripe me de manu pecca\textbf{tó}ris,~* et de manu contra legem agéntis \textit{et} \textit{in}\textbf{í}qui:
 & \textit{\small Deliver me, O my God, out of the hand of the sinner, * and out of the hand of the transgressor of the law and of the unjust.
}\\
6.\enspace Quóniam tu es patiéntia mea, \textbf{Dó}mine:~* Dómine, spes mea a juven\textit{tú}\textit{te} \textbf{me}a.
 & \textit{\small For thou art my patience, O Lord: * my hope, O Lord, from my youth.
}\\
7.\enspace In te confirmátus sum ex \textbf{ú}tero:~* de ventre matris meæ tu es pro\textit{téc}\textit{tor} \textbf{me}us.
 & \textit{\small By thee have I been confirmed from the womb: * from my mother’s womb thou art my protector.
}\\
8.\enspace In te cantátio mea semper:~\GreDagger\ tamquam prodígium factus sum \textbf{mul}tis:~* et tu ad\textit{jú}\textit{tor} \textbf{for}tis.
 & \textit{\small Of thee shall I continually sing: * I am become unto many as a wonder, but thou art a strong helper.
}\\
9.\enspace Repleátur os meum laude, ut cantem glóriam \textbf{tu}am:~* tota die magnitú\textit{di}\textit{nem} \textbf{tu}am.
 & \textit{\small Let my mouth be filled with praise, that I may sing thy glory; * thy greatness all the day long.
}\\
10.\enspace Ne projícias me in témpore senec\textbf{tú}tis:~* cum defécerit virtus mea, ne \textit{de}\textit{re}\textbf{lín}quas me.
 & \textit{\small Cast me not off in the time of old age: * when my strength shall fail, do not thou forsake me.
}\\
11.\enspace Quia dixérunt inimíci mei \textbf{mi}hi:~* et qui custodiébant ánimam meam, consílium fecé\textit{runt} \textit{in} \textbf{u}num.
 & \textit{\small For my enemies have spoken against me; * and they that watched my soul have consulted together,
}\\
12.\enspace Dicéntes: Deus derelíquit eum,~\GreDagger\ persequímini, et comprehéndite \textbf{e}um:~* quia non est \textit{qui} \textit{e}\textbf{rí}piat.
 & \textit{\small Saying: God hath forsaken him: pursue and take him, * for there is none to deliver him.
}\\
13.\enspace Deus ne elongéris \textbf{a} me:~* Deus meus, in auxílium \textit{me}\textit{um} \textbf{ré}spice.
 & \textit{\small O God, be not thou far from me: * O my God, make haste to my help.
}\\
14.\enspace Confundántur, et defíciant detrahéntes ánimæ \textbf{me}æ:~* operiántur confusióne et pudóre, qui quærunt \textit{ma}\textit{la} \textbf{mi}hi.
 & \textit{\small Let them be confounded and come to nothing that detract my soul; * let them be covered with confusion and shame that seek my hurt.
}\\
15.\enspace Ego autem semper spe\textbf{rá}bo:~* et adjíciam super omnem \textit{lau}\textit{dem} \textbf{tu}am.
 & \textit{\small But I will always hope; * and will add to all thy praise.
}\\
16.\enspace Os meum annuntiábit justítiam \textbf{tu}am:~* tota die salu\textit{tá}\textit{re} \textbf{tu}um.
 & \textit{\small My mouth shall shew forth thy justice; * thy salvation all the day long.
}\\
17.\enspace Quóniam non cognóvi litteratúram,~\GreDagger\ introíbo in poténtias \textbf{Dó}mini:~* Dómine, memorábor justítiæ tu\textit{æ} \textit{so}\textbf{lí}us.
 & \textit{\small Because I have not known learning, I will enter into the powers of the Lord: * O Lord, I will be mindful of thy justice alone.
}\\
18.\enspace Deus, docuísti me a juventúte \textbf{me}a:~* et usque nunc pronuntiábo mirabí\textit{li}\textit{a} \textbf{tu}a.
 & \textit{\small Thou hast taught me, O God, from my youth: * and till now I will declare thy wonderful works.
}\\
19.\enspace Et usque in senéctam et \textbf{sé}nium:~* Deus, ne \textit{de}\textit{re}\textbf{lín}quas me,
 & \textit{\small And unto old age and grey hairs: * O God, forsake me not,
}\\
20.\enspace Donec annúntiem bráchium \textbf{tu}um~* generatióni omni, \textit{quæ} \textit{ven}\textbf{tú}ra est:
 & \textit{\small Until I shew forth thy arm * to all the generation that is to come:
}\\
21.\enspace Poténtiam tuam, et justítiam tuam, Deus,~\GreDagger\ usque in altíssima, quæ fecísti ma\textbf{gná}lia:~* Deus, quis sí\textit{mi}\textit{lis} \textbf{ti}bi?
 & \textit{\small Thy power, and thy justice, O God, even to the highest great things thou hast done: * O God, who is like to thee?
}\\
22.\enspace Quantas ostendísti mihi tribulatiónes multas et malas:~\GreDagger\ et convérsus vivifi\textbf{cás}ti me:~* et de abýssis terræ íterum \textit{re}\textit{du}\textbf{xís}ti me:
 & \textit{\small How great troubles hast thou shewn me, many and grievous: and turning thou hast brought me to life, * and hast brought me back again from the depths of the earth:
}\\
23.\enspace Multiplicásti magnificéntiam \textbf{tu}am:~* et convérsus conso\textit{lá}\textit{tus} \textbf{es} me.
 & \textit{\small Thou hast multiplied thy magnificence; * and turning to me thou hast comforted me.
}\\
24.\enspace Nam et ego confitébor tibi in vasis psalmi veritátem \textbf{tu}am:~* Deus, psallam tibi in cíthara, \textit{Sanc}\textit{tus} \textbf{Is}raël.
 & \textit{\small For I will also confess to thee thy truth with the instruments of psaltery: * O God, I will sing to thee with the harp,
}\\
25.\enspace Exsultábunt lábia mea cum cantávero \textbf{ti}bi:~* et ánima mea, quam \textit{red}\textit{e}\textbf{mís}ti.
 & \textit{\small My lips shall greatly rejoice, when I shall sing to thee; * and my soul which thou hast redeemed.
}\\
26.\enspace Sed et lingua mea tota die meditábitur justítiam \textbf{tu}am:~* cum confúsi et revériti fúerint, qui quærunt \textit{ma}\textit{la} \textbf{mi}hi. & \textit{\small Yea and my tongue shall meditate on thy justice all the day; * when they shall be confounded and put to shame that seek evils to me.}\\
\end{longtable}

\gresetinitiallines{0}
\gregorioscore{versiculi/ferv-1}
Pater noster. \textit{secreto.}

\subsubsection*{Lectio I.}
\begin{tabular}{p{8cm} | p{8cm}}
Incipit Lamentátio Jeremíæ Prophétæ.
Aleph. Quómodo sedet sola cívitas plena pópulo: facta est quasi vídua dómina géntium: princeps provinciárum facta est sub tribúto.
Beth. Plorans plorávit in nocte, et lácrimæ ejus in maxíllis ejus: non est qui consolétur eam ex ómnibus caris ejus: omnes amíci ejus sprevérunt eam, et facti sunt ei inimíci.
Ghimel. Migrávit Judas propter afflictiónem, et multitúdinem servitútis: habitávit inter gentes, nec invénit réquiem: omnes persecutóres ejus apprehendérunt eam inter angústias.
Daleth. Viæ Sion lugent eo quod non sint qui véniant ad solemnitátem: omnes portæ ejus destrúctæ: sacerdótes ejus geméntes: vírgines ejus squálidæ, et ipsa oppréssa amaritúdine.
He. Facti sunt hostes ejus in cápite, inimíci ejus locupletáti sunt: quia Dóminus locútus est super eam propter multitúdinem iniquitátum ejus: párvuli ejus ducti sunt in captivitátem, ante fáciem tribulántis.
Jerúsalem, Jerúsalem, convértere ad Dóminum Deum tuum.
& \textit{Lesson from the book of Lamentations.
Aleph. How doth the city sit solitary that was full of people! how is the mistress of the Gentiles become as a widow: the princes of provinces made tributary!
Beth. Weeping she hath wept in the night, and her tears are on her cheeks: there is none to comfort her among all them that were dear to her: all her friends have despised her, and are become her enemies.
Ghimel. Juda hath removed her dwelling place because of her affliction, and the greatness of her bondage: she hath dwelt among the nations, and she hath found no rest: all her persecutors have taken her in the midst of straits.
Daleth. The ways of Sion mourn, because there are none that come to the solemn feast: all her gates are broken down: her priests sigh: her virgins are in affliction, and she is oppressed with bitterness.
He. Her adversaries are become her lords, her enemies are enriched: because the Lord hath spoken against her for the multitude of her iniquities: her children are led into captivity: before the face of the oppressor.
Jerusalem, Jerusalem, return to the Lord thy God.
}
\end{tabular}
\gresetinitiallines{1}
\gregorioscore{responsaria/ferv-1}

\subsubsection*{Lectio II.}
\begin{tabular}{p{8cm} | p{8cm}}
Vau. Et egréssus est a fília Sion omnis decor ejus: facti sunt príncipes ejus velut aríetes non inveniéntes páscua: et abiérunt absque fortitúdine ante fáciem subsequéntis.
Zain. Recordáta est Jerúsalem diérum afflictiónis suæ, et prævaricatiónis ómnium desiderabílium suórum, quæ habúerat a diébus antíquis, cum cáderet pópulus ejus in manu hostíli, et non esset auxiliátor: vidérunt eam hostes, et derisérunt sábbata ejus.
Heth. Peccátum peccávit Jerúsalem, proptérea instábilis facta est: omnes, qui glorificábant eam, sprevérunt illam, quia vidérunt ignomíniam ejus: ipsa autem gemens convérsa est retrórsum.
Teth. Sordes ejus in pédibus ejus, nec recordáta est finis sui: depósita est veheménter, non habens consolatórem: vide, Dómine, afflictiónem meam, quóniam eréctus est inimícus.
Jerúsalem, Jerúsalem, convértere ad Dóminum Deum tuum.
& \textit{Vau. And from the daughter of Sion all her beauty is departed: her princes are become like rams that find no pastures: and they are gone away without strength before the face of the pursuer.
Zain. Jerusalem hath remembered the days of her affliction, and prevarication of all her desirable things which she had from the days of old, when her people fell in the enemy's hand, and there was no helper: the enemies have seen her, and have mocked at her sabbaths.
Heth. Jerusalem hath grievously sinned, therefore is she become unstable: all that honoured her have despised her, because they have seen her shame: but she sighed and turned backward.
Teth. Her filthiness is on her feet, and she hath not remembered her end: she is wonderfully cast down, not having a comforter: behold, O Lord, my affliction, because the enemy is lifted up.
Jerusalem, Jerusalem, return to the Lord thy God.}
\end{tabular}
\gresetinitiallines{1}
\gregorioscore{responsaria/ferv-2}

\subsubsection*{Lectio III.}
\begin{tabular}{p{8cm} | p{8cm}}
Jod. Manum suam misit hostis ad ómnia desiderabília ejus: quia vidit gentes ingréssas sanctuárium suum, de quibus præcéperas ne intrárent in ecclésiam tuam.
Caph. Omnis pópulus ejus gemens, et quærens panem: dedérunt pretiósa quæque pro cibo ad refocillándam ánimam. Vide, Dómine, et consídera, quóniam facta sum vilis.
Lamed. O vos omnes, qui transítis per viam, atténdite, et vidéte, si est dolor sicut dolor meus: quóniam vindemiávit me, ut locútus est Dóminus in die iræ furóris sui.
Mem. De excélso misit ignem in óssibus meis, et erudívit me: expándit rete pédibus meis, convértit me retrórsum: pósuit me desolátam, tota die mæróre conféctam.
Nun. Vigilávit jugum iniquitátum meárum: in manu ejus convolútæ sunt, et impósitæ collo meo: infirmáta est virtus mea: dedit me Dóminus in manu, de qua non pótero súrgere.
Jerúsalem, Jerúsalem, convértere ad Dóminum Deum tuum.
& \textit{Jod. The enemy hath put out his hand to all her desirable things: for she hath seen the Gentiles enter into her sanctuary, of whom thou gavest commandment that they should not enter into thy church.
Caph. All her people sigh, they seek bread: they have given all their precious things for food to relieve the soul: see, O Lord, and consider, for I am become vile.
Lamed. O all ye that pass by the way, attend, and see if there be any sorrow like to my sorrow: for he hath made a vintage of me, as the Lord spoke in the day of his fierce anger.
Mem. From above he hath sent fire into my bones, and hath chastised me: he hath spread a net for my feet, he hath turned me back: he hath made me desolate, wasted with sorrow all the day long.
Nun. The yoke of my iniquities hath watched: they are folded together in his hand, and put upon my neck: my strength is weakened: the Lord hath delivered me into a hand out of which I am not able to rise.
Jerusalem, Jerusalem, return to the Lord thy God.}
\end{tabular}
\gresetinitiallines{1}
\gregorioscore{responsaria/ferv-3}

\end{document}